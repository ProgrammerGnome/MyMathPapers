\documentclass[12pt]{article}
    \title{\textbf{Formális nyelvek}}
    \author{Kidolgozott gyakorlati példák \\ (Szerkesztés alatt álló példatár)}
    \date{}
\usepackage[a4paper,
            left=0.5in,
            right=0.5in,
            top=0.5in,
            bottom=0.5in,
            footskip=.25in]{geometry}
\begin{document}

\maketitle

\section{Mely nyelveket generálják az alábbi grammatikák? \\ Adjuk meg a grammatikák típusait is!}


\subsection{Feladat: \\
$ G=< \{0,1\}, \{S,A,B\}, S, \{ S \rightarrow 01S|1A,\ A \rightarrow B0,\ B \rightarrow 1 \}> $}
\maketitle
\subsection{Megoldás}
A $ S \rightarrow 01S $ szabállyal a '01' stringek számát tudjuk növelni. Majd ha ezt meguntuk, 
az $ S \rightarrow (01)^n1A \vdash (01)^n1B0 \vdash (01)^n110 $ levezetés az egyetlen amivel folytatni tudjuk 
a generálást. Lezárni a generálást csak az $ B \rightarrow 0 $ szabály alkalmazásával tudjuk. \\\\
Tehát a keresett nyelv a következő:
$$ L = \{ \ (01)^n110, \ ahol \ n \in \mathbf{N}  \ \} $$
A G(L) pedig 2-es típusú, azaz környezetfüggetlen, mivel alakja: \\
$ A \rightarrow \omega \ , ahol \ \omega \in (V \cup W)^* , \ valamint \ A \in W^*. $

\subsection{Feladat: \\
$ G=< \{0,1\}, \{S,A,B\}, S, \{ S \rightarrow 0A,\ A \rightarrow 0A|1A|0B|1B,\ B \rightarrow 1B0|10 \}> $}
\maketitle
\subsection{Megoldás}
Kezdetben az $ S \rightarrow 0A $\ szabályt tudjuk csak alkalmazni, ezután viszont választhatunk A és B
generálási irány közül. 
Először válasszuk A irányt: $ 0A \vdash 00A| 01A \ ... $ használatával az első '0' string után
'0'-at vagy '1'-eseket generálhatunk tetszőleges sorrendben és mennyiségben, ez formalizálva így néz ki:
$ 0 \{0,1\}^* $. Ezek után a B irány maradt, ami először egyetlen darab '0'-t vagy '1'-et ad az eddigi string-hez.
A $ B \rightarrow 1B0 $ szabály a bal oldalon az '1'-esek, jobb oldalon a '0'-ák számát tudja növelni
azonos mértékben. Viszont zárásképp '10'-val zárhatunk, ami jó, hiszen ez nem borítja fel az imént leírt
szabályszerűséget. Így most tovább írhatjuk az eddig feltárt $ 0 \{0,1\}^* $ összefüggést:
$ 0 \{0,1\}^*1^m 0^m $. Ámde mert A-ból B-be el kell jutnunk valahogy amit csak egy '0' vagy '1' automatikus
hozzáadásával tehetünk meg, így a képlet $ \{0,1\}^* $ részénél ki kell vennünk az üres szót:
$ \{0,1\}^* \setminus \epsilon $. \\\\ 
Így a keresett nyelv végül a következőképp alakul:
$$ 0 \alpha 1^m 0^m \ | \ \alpha \in \{0,1\}^*\setminus\epsilon \ , \ m \in \mathbf{N}, \ m\geq 1 $$
A G(L) 2-es típusú grammatika. Indoklás a rész legelső feladatánál.


\section{Készítsünk reguláris grammatikákat az alábbi nyelvekhez!}
\subsection{Feladat: \\
$ L=\{ \alpha \in \{ 0,1 \}^* \ | \ \alpha-ra \ igaz \ X \} $ \\
Ahol X az, hogy: nem kezdődik egyessel vagy nem végződik nullára.}
\maketitle
\subsection{Megoldás}
Először is nézzük végig a lehetséges eseteket: 0x0; 0x1; 1x0; 1x1
(itt az x bármilyen és bármennyi véges V*-beli string-et jelöl).
A 'vagy'-al öszekötött állítások igazságértéke csak akkor hamisak ha mindkét állítás hamis.
Ebből az adódik, hogy 0x1 kivételével az összes eset megengedett. \\\\
A grammatika tehát így néz ki: \\
$$ G=<\{ \{0,1\}, \{S,A,B \}, S, \{ S \rightarrow 0|1|1A|0B, \ A \rightarrow 0|1|0A|1A, \ B 
\rightarrow 0B|1B|1 \} \}> $$
A működésének magyarázata esetszétválasztással egyszerűen elintézhető. A kezdőszimbólumból való távozások
közül az B jelöli, hogy egyessel csak végezhetünk ($ B \rightarrow 1 $), 
A pedig hogy bármivel ($ A \rightarrow 0|1 $).
Az $ A \rightarrow 0A|1A $ pedig azt biztosítja, hogyha a kezdő -és zárószimbólum közt bármilyen 
string lehessen, ugyanígy a $ B \rightarrow 0B|1B $
esetnél is.

\subsection{Feladat: \\
$ L=\{ \alpha \in \{ a,b \}^* \ | \ \alpha-ra \ igaz \ X \} $ \\
Ahol X az, hogy: első és utolsó betűje különbözik.}
\maketitle
\subsection{Megoldás}
Először is nézzük végig a lehetséges eseteket: axa; axb; bxa; bxb. Ebből rögtön ki tudjuk szűrni mi kell nekünk:
axb és bxa (itt az x bármilyen és bármennyi véges V*-beli string-et jelöl).
Ezek alapján az előző feladat mintájára tudjuk megkonstruálni a grammatikánkat:
$$ G=<\{ \{a,b\}, \{S,A,B \}, S,
\{ S \rightarrow aB|bA|\epsilon, \ B \rightarrow aB|bB|b, \ A \rightarrow aA|bA|a. \} \}> $$
A működésének magyarázata igen egyszerű és hasonló az előző feladatéhoz:
esetszétválasztást használunk itt is. A B irány it a a-val való kezdést és b-vel való zárást takarja.
Az A irány pedig a fordítottját, azaz a b-vel való kezdést és a-val történő lezárást.
Az $ A \rightarrow aA|bA $ pedig azt biztosítja, hogyha a kezdő -és zárószimbólum közt bármilyen 
string lehessen, ugyanígy a $ B \rightarrow aB|bB $ esetnél is.

\section{Készítsünk (bármilyen) grammatikákat az alábbi nyelvekhez!}
\subsection{Feladat: \\
$ L=\{ a^{2n}b^{3n}ab, \ n \in \mathbf{N} \} $}
\maketitle
\subsection{Megoldás}
Ezt a feladatot sajnos nem tudjuk reguláris grammatikával megoldani, amit készíteni fogunk az \\
környezetfüggetlen lesz. Az ötletünk a következő: $ S \rightarrow aaAbbbab, \ A \rightarrow aaAbbb|aabbb $.
Ezzel először S-ből átmegyünk A-ba, ami azért van mert nem szeretnénk a szó végi 'ab' generálását a szó
belsejében ismételgetni. A-ból viszont véges sokszor tudjuk a szó belsejében az 'aabbb'-k generálását
ismételgetni a $ A \rightarrow aaAbbb $ szabály alkalmazásával. Ha mindezt meguntuk, akkor
a $ A \rightarrow aabbb $ szabállyal zárunk. De ne feledkezzünk meg arról sem, hogy az 'aabbbab' és 'ab'
szavakat is tudnia kell generálnia a grammatikánknak, ezt a $ S \rightarrow aabbbab|ab $ szabályok biztosítják.
Így végül a következő grammatika lesz a megoldás:
$$ G=<\{ \{a,b \}, \{S,A \}, S, \{ S \rightarrow aaAbbbab|aabbbab|ab, \ A \rightarrow aaAbbb|aabbb \} \}> $$

\subsection{Feladat: \\
$ L=\{ 10^{3k}1^{2k}0, \ k \in \mathbf{N} \} $}
\maketitle
\subsection{Megoldás}
A megoldásnál itt is hasonlóan járunk el, mint az előző esetben, szintén környezetfüggetlen grammatikát gyártunk.
Az ötletünk a következő: $ S \rightarrow 000A11, \ A \rightarrow 000A11|00011 $.
Ezzel először S-ből átmegyünk A-ba, ami azért van mert nem szeretnénk a szó eleji '1' és szóvégi '0' 
generálását a szó belsejében ismételgetni. A-ból viszont véges sokszor tudjuk a szó belsejében az '00011'-k
generálását ismételgetni a $ A \rightarrow 00A11 $ szabály alkalmazásával. Ha mindezt meguntuk, akkor
a $ A \rightarrow 00011 $ szabállyal zárunk. De ne feledkezzünk meg arról sem, hogy az '1000110' és '10'
szavakat is tudnia kell generálnia a grammatikánknak, ezt a $ S \rightarrow 1000110|10 $ szabályok biztosítják.
Így végül a következő grammatika lesz a megoldás:
$$ G=<\{ \{0,1 \}, \{S,A \}, S, \{ S \rightarrow 1000A110|1000110|10, \ A \rightarrow 000A11|00011 \} \}> $$

\section{Formális nyelvek metszete, tükörképe és iteráltja}
\subsection{Feladat: \\
$ Legyen \ L_{1}=\{ ab^{n} \ : \ n \in \mathbf{N} \}, \ L_{2}=\{ a^{n}b \ : \ n \in \mathbf{N} \} .$ \\
a) $ L_{1} \cup L_{2} = ?, \ valamint\ L_{1} \cap L_{2} = ? $ \\
b) $ Az \ L_{1}^{*} = \{\alpha^{-1} \ : \ \alpha \in L_{2} \} \ jelolessel, \ L_{1}*L_{2}^{-1}=? $ \\
c) $ Igaz-e, \ hogy \ L_{1}^{*}=L_{2}^{*} \ ? \ Ha\ nem,\ miert?\ Ha\ igen\,\ mi\ ez\ a\ nyelv? $}
\maketitle
\subsection{Megoldás}
Az a) feladatrészre a megoldás a következő: $ L = \{ ab^{n}a^{m}b, \ n,m \in \mathbf{N} \}$, 
mivel a formális nyelvek metszete nem más
mint az összeszorzásuk, ez utóbbi pedig konkatenálást (egymás után illesztést) jelent. \\
A metszet az $ L=\{ ab \} $ lesz, mivel az az az egy szó, ami mindkét nyelvben előfordul. \\\\
A b) feladatrészre a válaszunk: $ L = \{ ab^{n}ba^{m} \ n,m \in \mathbf{N} \} $.
Mivel itt az $ \alpha^{-1} $-en nem jelent mást, mint a $ \alpha $ szó tükörképét, ami valójában az őt alkotó
string-ek fordított sorrendben való egymás utáni leírását takarja. \\\\
Végezetül a c) feladarészhez érkeztünk. Az $ L^{*} = \bigcup_{n \geq 0} L^{n} $ összefüggés jelentését
kell itt tulajdonképp tárgyalnunk. Tehát L nyelv iteráltja nem más, mint a szavainak tetszőleges véges sok
tényezőből álló szorzatainak halmaza. Visszakanyarodva a kérdésünkhöz: az egyenlőség nem teljesül. \\
Indoklás: mivel $ L_{1}=\{ a, ab, abb, abbb, ... \} $ és $ L_{2}=\{ b, ab, aab, aaab, ... \} $ nyelvek szavainak
szorzata között $L_{2}$ esetén előállhat pl. 'bab', de az $L_{1}$ nyelvnél ez nem lehetséges, tehát találtunk
egy ellenpéldát.

\section{Két nyelv metszete és uniója}
\subsection{Feladat: \\
Készítsünk két grammatikát, aminek az uniója az alábbi \\ nyelvet generálja:
$ L=\{ ba^{n}ab^{m}ba \ : \ n,m \in \mathbf{N} \} $}
\maketitle
\subsection{Megoldás}
A nyelvet két részre szedjük és mindkét "részéhez" készítünk egy reguláris grammatikát, majd vesszük a két
grammatika unióját. Megjegyezzük, hogy két grammatika uniója is reguláris kell legyen.
A felosztás nézzen ki a következőképp: $ L_{1}=\{ ba^{n}a \ ...\} $ és $ L_{2}=\{ b^{n}ba \ ...\} $.
Itt fontos megemlíteni, hogy az n hatvány az nyelvenként van érvényben. Tehát az únióban majd végül n és m
fog szerepelni.
Az $ G(L_{1})=\{ ..., \ S \rightarrow bA, A \rightarrow aA|a \} $ generálja, a
$ G(L_{2})=\{ ..., \ S \rightarrow bS|bA, A \rightarrow a \} $ generálja. Ennek a kettőnek a működése
triviális, azokat most terjedelmi okok miatt nem részletezzük. Végül vesszük a két nyelv unióját, ami
valójában a szorzatukat (azaz magyarán egymás után helyezésüket, konkatenálásukat) jelenti.
Ezt úgy érjük el, hogy először a két nyelv terminális jeleinek halmazait diszjunktá tesszük, ez valójában
a "nagybetűk bevesszőzése".
Majd az $L_{1}$ nyelvben a zárószimbólumok végét kiegészítjük a $L_{2}$ grammatika
kezdőszimbólumával, az egyik nyelvből a másikba való átmenet végett. \\\\
Itt megemlítjük azt, hogy ebben a feladatban az $\epsilon$ sehol sem szerepelt a kezdőszimbólumokból kiindulva,
így most ezzel nem kell vacakolnunk. De egyébként ha lenne ilyen arra is kitérünk: \\
Tegyük fel, hogy az $ L_{1} $-ben az $S \rightarrow bA$ helyett az $S \rightarrow bA|\epsilon$ 
szabály szerepel, ekkor $\epsilon$-t csak egyszerűen kivesszük az unióból és helyére az "átvivőszabályt" 
írjuk be, ami jelen esetben az $A \rightarrow aS^{'}$. Ez biztosítja, hogy "semmi" generálás után rögtön a
második nyelv elejére jutunk. Ha az $L_{1}$ és $L_{2}$ nyelvben egyszerre szerepel az $\epsilon$,
akkor egyszerűen az új kezdőszimbólum jobb oldalára be kell írni az $\epsilon$-t is.
Ezek után viszont figyeljünk rá, hogy az új kezdőszimbólum nem fordul elő-e a szabályok jobb oldalán. \\\\
Visszatérve az eredeti feladatunkhoz a megoldás a következőképp alakul: \\
$$ G=<\{ \{a,b\}, \{S,S^{'},A,A^{'},Q \}, Q, \{ Q \rightarrow bA, \ A \rightarrow aA|aS^{'}, \ S^{'} 
\rightarrow bS^{'}|bA^{'}, \ A^{'} \rightarrow a \} \}> . $$

\subsection{Feladat: \\
Készítsünk két grammatikát, aminek a metszete az alábbi \\ nyelvet generálja:
$ L=\{ \alpha ab^{n}b \ |\ \alpha \in \{ \epsilon, a, b \} \ , \ n \in \mathbf{N} \} $}
\maketitle
\subsection{Megoldás}
A nyelvet két részre szedjük és mindkét "részéhez" készítünk egy reguláris grammatikát, majd vesszük a két
nyelv metszetét, ami az összeadásukat jelenti. Az összeadás itt valami olyasmit takar, hogy egyik vagy
másik nyelv generálható, hozzátéve hogy mindkét nyelv nem generálható egyszerre, de legalább az egyik
mindeképp generálható. \\
Az $L_{1}=\{ ..., \ aab^{n}b \}$ legyen, az $L_{2}=\{ ..., \ bab^{n}b \}$. Ezek után az $L_{1}$ és $L_{2}$-höz
késztünk grammatikákat, úgy hogy a kezdő szabályokhoz még egy $\epsilon$-t is beveszünk. \\
A követkőképpen néznek ki:
$G(L_{1})=\{ ..., \ S \rightarrow aA|\epsilon,\ A \rightarrow aB,\ B \rightarrow bB|b \}$, valamint
$G(L_{2})=\{ ..., \ S^{'} \rightarrow bA^{'}|\epsilon,\ A^{'} \rightarrow aB^{'},\ B^{'} \rightarrow bB^{'}|b \}$.
Már a terminális jelek diszjunktá tétele is megtörtént.
Ezek után nincs más dolgunk, minthogy bevezetünk egy új kezdőállapotot, amiből elérhetővé tesszük az
$L_{1}$ és $L_{2}$ béli szavakat is. Tehát $G(L)=\{..., \ Q \rightarrow aA|bA^{'}|\epsilon \}$, majd ezek után
az összes szabályt változatlanul bemásoljuk:
$G(L)=\{..., \ Q \rightarrow aA|bA^{'}|\epsilon, A \rightarrow aB, B \rightarrow bB|b, A^{'} \rightarrow aB^{'},
B^{'} \rightarrow bB^{'}|b \}$. Mivel $Q \rightarrow \epsilon$ szerepel, így át kell néznük, hogy Q nem-e fordul
elő a szabályok jobb oldalán, de nem. Jók vagyunk! A teljes megoldás így írható fel:
$$ G(L)=<\{ \{ a,b \}, \{ S,S^{'},A,A^{'},B,B^{'},Q \}, \ Q, \{ Q \rightarrow aA|bA^{'}|\epsilon, A \rightarrow 
aB, B \rightarrow bB|b, A^{'} \rightarrow aB^{'}, B^{'} \rightarrow bB^{'}|b \}\}> $$

\section{A formális nyelvek mint halmazok}
\subsection{Feladatok és megoldásai: \\}
$ V=\{ a,b,c \}, \ W=\{ c,d,e \} .$ \\\\
$ V*(W \setminus V)*W^* \ =\ \{a,b,c\}*\{d,e\}*\{\epsilon, c, d, e, cd, ec, ed, cde, ecd, ...\} $ \\
$= \{ad, ae, bd, be, cd, ce\}*\{\epsilon, c, d, e, cd, ec, ed, cde, ecd, ...\} = \{ ad\alpha,
ae\alpha, bd\alpha, be\alpha, cd\alpha, ce\alpha \} $, ahol $\alpha \in \{c,d,e \}^{*}$ \\\\
$ (V \setminus W)(W \setminus V) \ =\ \{a,b\}*\{d,e\} = \{ ad, ae, bd, be \} $ \\
$ V^* \setminus W^*\ \Longrightarrow \ $ Minden lehet vélgülis, csak az $\epsilon$-okat
és a 'c'-t vettük ki. Tehát minden $\{ a,b,c,d,e \} $-t tartalmazó szó kijöhet, ami nem tartalmaz
$\epsilon$-t és nem csak 'c'-ből áll (c lehet benne).






\end{document}
